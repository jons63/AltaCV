% docker run -ti -v miktex:/miktex/.miktex -v C:/git/AltaCV:/miktex/work miktex/miktex pdflatex jongar.tex
%%%%%%%%%%%%%%%%%
% This is an sample CV template created using altacv.cls
% (v1.3, 10 May 2020) written by LianTze Lim (liantze@gmail.com). Now compiles with pdfLaTeX, XeLaTeX and LuaLaTeX.
%
%% It may be distributed and/or modified under the
%% conditions of the LaTeX Project Public License, either version 1.3
%% of this license or (at your option) any later version.
%% The latest version of this license is in
%%    http://www.latex-project.org/lppl.txt
%% and version 1.3 or later is part of all distributions of LaTeX
%% version 2003/12/01 or later.
%%%%%%%%%%%%%%%%

%% If you are using \orcid or academicons
%% icons, make sure you have the academicons
%% option here, and compile with XeLaTeX
%% or LuaLaTeX.
% \documentclass[10pt,a4paper,academicons]{altacv}

%% Use the "normalphoto" option if you want a normal photo instead of cropped to a circle
% \documentclass[10pt,a4paper,normalphoto]{altacv}

\documentclass[10pt,a4paper,ragged2e,withhyper]{altacv}
%% AltaCV uses the fontawesome5 and academicons fonts
%% and packages.
%% See http://texdoc.net/pkg/fontawesome5 and http://texdoc.net/pkg/academicons for full list of symbols. You MUST compile with XeLaTeX or LuaLaTeX if you want to use academicons.

% Change the page layout if you need to
\geometry{left=1.25cm,right=1.25cm,top=1.5cm,bottom=1.5cm,columnsep=1.2cm}

% The paracol package lets you typeset columns of text in parallel
\usepackage{paracol}

% Change the font if you want to, depending on whether
% you're using pdflatex or xelatex/lualatex
\ifxetexorluatex
  % If using xelatex or lualatex:
  \setmainfont{Roboto Slab}
  \setsansfont{Lato}
  \renewcommand{\familydefault}{\sfdefault}
\else
  % If using pdflatex:
  \usepackage[rm]{roboto}
  \usepackage[defaultsans]{lato}
  % \usepackage{sourcesanspro}
  \renewcommand{\familydefault}{\sfdefault}
\fi

% Change the colours if you want to
\definecolor{SlateGrey}{HTML}{2E2E2E}
\definecolor{LightGrey}{HTML}{666666}
\definecolor{DarkPastelRed}{HTML}{450808}
\definecolor{PastelRed}{HTML}{8F0D0D}
\definecolor{GoldenEarth}{HTML}{E7D192}
\colorlet{name}{black}
\colorlet{tagline}{PastelRed}
\colorlet{heading}{DarkPastelRed}
\colorlet{headingrule}{GoldenEarth}
\colorlet{subheading}{PastelRed}
\colorlet{accent}{PastelRed}
\colorlet{emphasis}{SlateGrey}
\colorlet{body}{LightGrey}

% Change some fonts, if necessary
\renewcommand{\namefont}{\Huge\rmfamily\bfseries}
\renewcommand{\personalinfofont}{\footnotesize}
\renewcommand{\cvsectionfont}{\LARGE\rmfamily\bfseries}
\renewcommand{\cvsubsectionfont}{\large\bfseries}


% Change the bullets for itemize and rating marker
% for \cvskill if you want to
\renewcommand{\itemmarker}{{\small\textbullet}}
\renewcommand{\ratingmarker}{\faCircle}

%% sample.bib contains your publications
\addbibresource{sample.bib}

\begin{document}
\name{Jonas Garsten}
\tagline{Embedded software engineer}
%% You can add multiple photos on the left or right
\photoR{2.8cm}{meNo2.png}
% \photoL{2.5cm}{Yacht_High,Suitcase_High}

\personalinfo{%
  % Not all of these are required!
  \email{jonasgarsten@gmail.com}
  \phone{0707-75-73-79}
  \mailaddress{20, Halvsekelsgatan, 41535 Gothenburg}
  \location{Gothenburg, Sweden}
  \homepage{www.homepage.com}
  %\twitter{@twitterhandle}
  \linkedin{jongar}
  \github{jons63}
  %% You MUST add the academicons option to \documentclass, then compile with LuaLaTeX or XeLaTeX, if you want to use \orcid or other academicons commands.
  % \orcid{0000-0000-0000-0000}
  %% You can add your own arbtrary detail with
  %% \printinfo{symbol}{detail}[optional hyperlink prefix]
  % \printinfo{\faPaw}{Hey ho!}[https://example.com/]
  %% Or you can declare your own field with
  %% \NewInfoFiled{fieldname}{symbol}[optional hyperlink prefix] and use it:
  % \NewInfoField{gitlab}{\faGitlab}[https://gitlab.com/]
  % \gitlab{your_id}
}

\makecvheader
%% Depending on your tastes, you may want to make fonts of itemize environments slightly smaller
% \AtBeginEnvironment{itemize}{\small}

%% Set the left/right column width ratio to 6:4.
\columnratio{0.6}

% Start a 2-column paracol. Both the left and right columns will automatically
% break across pages if things get too long.
\begin{paracol}{2}
\cvsection{Experience}

\cvevent{Development Engineer}{QRTECH AB}{September 2019 -- Ongoing}{Mölndal, Sweden}
\begin{itemize}
\item Worked on Volvo Cars climate control system for the SPA2 platform. The project is 
built on NXPs S32K14x platform for which I helped develop an application using Vectors 
DaVinci platform and a bootloader. 
\item Developing internal processes for CI/CD and release management. 
\end{itemize}

\cvsection{About me}

I am someone who easily integrates into new environments and get along with anyone. 
It is always fun to meet and work with new people from which I can learn something from.
Preferably \newline assignments shall include topics that I want to learn. This is not a big hurdle
to clear as I am interested in everything from web development to circute board design. 
If I had to choose what topics I am most interested in learning, I would say C++, 
circute board design and bluetooth as these are topics I will have to learn to realise 
some projcet ideas of mine.

\medskip

% use ONLY \newpage if you want to force a page break for
% ONLY the current column
%\newpage

%% Switch to the right column. This will now automatically move to the second
%% page if the content is too long.
\switchcolumn

\cvsection{Most Proud of}

\cvachievement{\faTrophy}{Masters Degree}{Finishing my thesis while working full time}

\divider

\cvachievement{\faBatteryFull}{Investing in Myself}{Finding the time and energy for personal projects}

\divider

\cvachievement{\faFire}{Finding my Passion}{Figuring out that coding is what I burn for}

\cvsection{Strengths}

\cvtag{Hard-working}
\cvtag{Team player}\\
\cvtag{Hunger to learn}

\divider\smallskip

\cvtag{C}
\cvtag{Embedded Systems}\\
\cvtag{Scripting}

\cvsection{Languages}

\cvskill{Swedish}{5}
\divider

\cvskill{English}{5}

%% Yeah I didn't spend too much time making all the
%% spacing consistent... sorry. Use \smallskip, \medskip,
%% \bigskip, \vpsace etc to make ajustments.
\medskip

\cvsection{Education}

\cvevent{M.Sc.\ Systems, Control and \newline Mechatronics}{Chalmers University of Technology}{Sept 2017 -- Jan 2020}{}

\divider

\cvevent{B.Sc.\ in Automation and Mechatronics}{Chalmers University of Technology}{Sept 2014 -- June 2017}{}

% \divider

\cvsection{Referees}

% \cvref{name}{email}{mailing address}
Can be arranged on request


\end{paracol}


\end{document}
